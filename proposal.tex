\documentclass{article}
\usepackage[utf8]{inputenc}
\usepackage{amsmath}
\usepackage{amssymb}
\usepackage{mathtools}
\usepackage[fleqn]{nccmath}
\usepackage{enumitem}
\usepackage{indentfirst}
\usepackage{graphicx}
\usepackage{caption}
\usepackage{subcaption}
\usepackage{color} %red, green, blue, yellow, cyan, magenta, black, white
\usepackage[dvipsnames]{xcolor}
\usepackage{tikz,pgfplots,tikz-3dplot}
\usepackage{listings}
\usepackage{cancel}
\usepackage{braket}
\usepackage{esvect,esint}
\usepackage{float}
\usepackage{bm}
\usepackage{mdframed}
\usepackage{multirow}
\usepackage{multicol}
\usepackage{hyperref}

\usepackage{titling}
\usepackage{titlesec} % Section headings
\titleformat*{\section}{\large\bfseries}
\titleformat*{\subsection}{\normalsize\bfseries}
\titleformat*{\subsubsection}{\normalsize\bfseries}
\titlespacing{\section}{0pt}{10pt}{0pt}
\titlespacing{\subsection}{0pt}{10pt}{0pt}
\titlespacing{\subsubsection}{0pt}{10pt}{0pt}

\usepackage{geometry}
\geometry{
    a4paper,
    margin=1in, % Margin size, obviously.
    top = 1in,
    bottom = 0.5in,
}

\usepackage{fancyhdr}
\pagestyle{fancy}
\fancyhf{}
\lhead{Ethan Pham}
\rhead{\thepage}

\usetikzlibrary{shapes, calc, arrows, arrows.meta, patterns, decorations.markings, decorations.pathmorphing, patterns.meta}

\def\centerarc[#1](#2)(#3:#4:#5)% Syntax: [draw options] (center) (initial angle:final angle:radius)
    { \draw[#1] ($(#2)+({#5*cos(#3)},{#5*sin(#3)})$) arc (#3:#4:#5); }

\usepackage{color} %red, green, blue, yellow, cyan, magenta, black, white
\graphicspath{ {./Images/} }

\definecolor{string}{RGB}{230, 219, 116}
\definecolor{comment}{RGB}{117, 113, 94}
\definecolor{normal}{RGB}{0,0,0}
\definecolor{identifier}{RGB}{166, 226, 46}

\lstset{
	language=python,                			% choose the language of the code
  	numbers=left,                   		% where to put the line-numbers
  	stepnumber=1,                   		% the step between two line-numbers.        
  	numbersep=5pt,                  		% how far the line-numbers are from the code
  	numberstyle=\tiny\color{black}\ttfamily,
  	showspaces=false,               		% show spaces adding particular underscores
  	showstringspaces=false,         		% underline spaces within strings
  	showtabs=false,                 		% show tabs within strings adding particular underscores
  	tabsize=4,                      		% sets default tabsize to 2 spaces
  	captionpos=b,                   		% sets the caption-position to bottom
  	breaklines=true,                		% sets automatic line breaking
  	title=\lstname,                 		% show the filename of files included with \lstinputlisting;
  	basicstyle=\color{normal}\ttfamily,					% sets font style for the code
  	keywordstyle=\color{magenta}\ttfamily,	% sets color for keywords
  	stringstyle=\color{string}\ttfamily,		% sets color for strings
  	commentstyle=\color{comment}\ttfamily,	% sets color for comments
  	emph={format_string, eff_ana_bf, permute, eff_ana_btr},
  	emphstyle=\color{identifier}\ttfamily
}

% Calculus Commands
\newcommand{\infint}{\int_{-\infty}^{\infty}}
\newcommand{\fd}[2]{\dfrac{\mathrm{d}^{#2}}{\mathrm{d}#1^{#2}}} % For full derivatives. \fd{variable}{number}
\newcommand{\ffd}[3]{\dfrac{\mathrm{d}^{#3}#1}{\mathrm{d}#2^{#3}}} % For full derivatives with function. \ffd{function}{variable}{number}
\newcommand{\pd}[2]{\dfrac{\partial^{#2}}{\partial #1^{#2}}} % For partial derivatives. \pd{variable}{number}
\newcommand{\fpd}[3]{\dfrac{\partial^{#3}#1}{\partial #2^{#3}}} % For partial derivatives with function. \fpd{function}{variable}{number}


% Number symbols
\newcommand{\R}{\mathbb{R}}
\newcommand{\N}{\mathbb{N}}
\newcommand{\Z}{\mathbb{Z}}
\newcommand{\Q}{\mathbb{Q}}

% Vector Commands
\renewcommand{\vv}[1]{\bm{\mathrm{#1}}}
\newcommand{\uvv}[1]{\hat{\bm{\mathrm{#1}}}}
\newcommand{\dvv}[1]{\dot{\bm{\mathrm{#1}}}}
\newcommand{\ddvv}[1]{\ddot{\bm{\mathrm{#1}}}}
\newcommand{\grad}{\bm{\mathrm{\nabla}}}
\newcommand{\abs}[1]{\left|#1\right|}

% Trig Commands
\newcommand{\arccot}{\operatorname{arccot}}
\newcommand{\csch}{\operatorname{csch}}
\newcommand{\sech}{\operatorname{sech}}

% QED
\newcommand{\qed}{\tag*{\(\square\)}}

% Typing Commands
\newcommand{\note}[1]{\color{red}#1\color{black}}


\pgfplotsset{compat=1.18}
\usepgfplotslibrary{colorbrewer}

\begin{document}
\section*{Graviational Lensing for a Schwarzschild Black Hole using Backwards Ray Tracing}
	General relativity says that matter tells space how to bend, and space tells matter how to move. Assuming a cosmological constant of zero, this relation is governed by the Einstein field equations:
	\begin{align}
		R_{\mu\nu}-\frac{1}{2}R=8\pi G T_{\mu\nu}
	\end{align}
	The Schwarzschild solution is the simplest case; a completely stationary mass centered at the origin, whose metric (using \(-,+,+,+\)) in spherical coordinates is well known to be
	\begin{align}
		\mathrm{d}s^{2}=-\left(1-\frac{r_{s}}{r}\right)\mathrm{d}t^{2}+\left(1-\frac{r_{s}}{r}\right)^{-1}\mathrm{d}r^{2}+r^{2}\mathrm{d}\theta^{2}+r^{2}\sin^{2}\theta\mathrm{d}\phi^{2}
	\end{align}
	The spherical symmetry of the problem allows us to calculate the geodesics in a plane (\(\theta=\pi/2\)). We can write a pair of differential equations for \(r\) and \(\phi\) in terms of the conserved quantities energy \(E\) and angular momentum \(L\):
	\begin{align}\label{eq:eom}
		\begin{aligned}
			\ddot{\phi}&=\frac{L}{r^{2}}, & \ddot{r}&=-\frac{R_{s}}{2r^{2}}\left(\frac{L^{2}}{r^{2}}+\epsilon\right)+\frac{L^{2}}{r^{3}}\left(1-\frac{R_{s}}{r}\right)
		\end{aligned}
	\end{align}
	Where the quantity \(\epsilon=0\) since light travels on null geodesics. These equations can be solved numerically on a computer to generate the path that light takes in the presence of a Schwarzschild black hole. 
	
\subsection*{Approach}
Figuring out the photon trajectories comes down to integrating the equations of motion \ref{eq:eom} with the Runge-Kutta-Fehlberg method for the dynamic time step. The energy \(E\) and angular momentum \(L\) are both determined by initial conditions. To make a lensed image, we start with an original image and a plane representing the final, distorted image observed on opposite sides of the black hole. Then, we imagine a ray, representing a photon, hits the plane at some point/pixel while perpendicular to the plane. We then trace the ray backwards from this pixel. When the ray goes backwards, there are 3 possible results:
\begin{enumerate}
	\item It enters the black hole, in which case, the pixel will be set to black.
	\item  It shoots off to infinity, in which case, the pixel will be set to black. Obviously we can't simulate it for infinite time, but this is the physical idea represented by the ray going out of whatever bounds we choose.
(Note: While debugging, we may set the pixel to a different color depending on whether it enters the black hole or goes out of bounds.)
	\item It hits a pixel on the original image, in which case, the pixel on the plane which we were tracing back from is set to be the same color as the pixel which the pixel of the original image which the backtraced ray hit.
\end{enumerate}
This, in theory, should allow us to get the same distortion effects that are obtained using lensing equations and angles of deflection (our approach seems to be the most common way to do this). 

\subsection*{Objectives}
\begin{enumerate}
	\item Simulate and render a 3D trajectory for a single photon.
	\item Check simulation by using known solutions (i.e. ICs where light obrits the black hole are known, so setting those in the simulation should match).
	\item Generate an image displaying the lensing effect using a ``background galaxy" (which will probably be an image off of Google). Or if this is too computationally expensive, use a lower resolution image. 
	\item Compare the results with other (physically correct) lensing simulations that use the angle of deflection and see how well it matches.
	\item If time allows, translate or tilt the image it at an angle to simulate changing the perspective of the lensing and make a short video displaying the lensing effect.
\end{enumerate}

	
	
	
	
\end{document}